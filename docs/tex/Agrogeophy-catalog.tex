%% Generated by Sphinx.
\def\sphinxdocclass{report}
\documentclass[letterpaper,10pt,english]{sphinxmanual}
\ifdefined\pdfpxdimen
   \let\sphinxpxdimen\pdfpxdimen\else\newdimen\sphinxpxdimen
\fi \sphinxpxdimen=.75bp\relax

\PassOptionsToPackage{warn}{textcomp}
\usepackage[utf8]{inputenc}
\ifdefined\DeclareUnicodeCharacter
% support both utf8 and utf8x syntaxes
  \ifdefined\DeclareUnicodeCharacterAsOptional
    \def\sphinxDUC#1{\DeclareUnicodeCharacter{"#1}}
  \else
    \let\sphinxDUC\DeclareUnicodeCharacter
  \fi
  \sphinxDUC{00A0}{\nobreakspace}
  \sphinxDUC{2500}{\sphinxunichar{2500}}
  \sphinxDUC{2502}{\sphinxunichar{2502}}
  \sphinxDUC{2514}{\sphinxunichar{2514}}
  \sphinxDUC{251C}{\sphinxunichar{251C}}
  \sphinxDUC{2572}{\textbackslash}
\fi
\usepackage{cmap}
\usepackage[T1]{fontenc}
\usepackage{amsmath,amssymb,amstext}
\usepackage{babel}



\usepackage{times}
\expandafter\ifx\csname T@LGR\endcsname\relax
\else
% LGR was declared as font encoding
  \substitutefont{LGR}{\rmdefault}{cmr}
  \substitutefont{LGR}{\sfdefault}{cmss}
  \substitutefont{LGR}{\ttdefault}{cmtt}
\fi
\expandafter\ifx\csname T@X2\endcsname\relax
  \expandafter\ifx\csname T@T2A\endcsname\relax
  \else
  % T2A was declared as font encoding
    \substitutefont{T2A}{\rmdefault}{cmr}
    \substitutefont{T2A}{\sfdefault}{cmss}
    \substitutefont{T2A}{\ttdefault}{cmtt}
  \fi
\else
% X2 was declared as font encoding
  \substitutefont{X2}{\rmdefault}{cmr}
  \substitutefont{X2}{\sfdefault}{cmss}
  \substitutefont{X2}{\ttdefault}{cmtt}
\fi


\usepackage[Bjarne]{fncychap}
\usepackage[,numfigreset=1,mathnumfig]{sphinx}

\fvset{fontsize=\small}
\usepackage{geometry}


% Include hyperref last.
\usepackage{hyperref}
% Fix anchor placement for figures with captions.
\usepackage{hypcap}% it must be loaded after hyperref.
% Set up styles of URL: it should be placed after hyperref.
\urlstyle{same}

\addto\captionsenglish{\renewcommand{\contentsname}{Contents:}}

\usepackage{sphinxmessages}
\setcounter{tocdepth}{1}



\title{Agrogeophy-catalog Documentation}
\date{Dec 17, 2020}
\release{0.1.0}
\author{agrogeophy}
\newcommand{\sphinxlogo}{\vbox{}}
\renewcommand{\releasename}{Release}
\makeindex
\begin{document}

\pagestyle{empty}
\sphinxmaketitle
\pagestyle{plain}
\sphinxtableofcontents
\pagestyle{normal}
\phantomsection\label{\detokenize{index::doc}}



\chapter{Getting Started}
\label{\detokenize{getting-started:getting-started}}\label{\detokenize{getting-started::doc}}
\sphinxhref{https://agrogeophy.slack.com/}{\sphinxincludegraphics{{/home/runner/work/catalog/catalog/docs/tex/.doctrees/images/9ca7c69c7a888bdf4340e21de765bed9e1a8b192/Slack-agrogeophy-1}.svg}}

\sphinxhref{https://doi.org/10.5281/zenodo.4058524}{\sphinxincludegraphics{{/home/runner/work/catalog/catalog/docs/tex/.doctrees/images/0a96cfbd8573408ea7e006027cb81d65ecfa3622/zenodo.4058524}.svg}}


\section{About}
\label{\detokenize{getting-started:about}}
\sphinxstylestrong{Agrogeophy\sphinxhyphen{}catalog is part of the agrogeophysical catalog website:} \sphinxurl{http://geo.geoscienze.unipd.it/growingwebsite/map\_catalog}

We welcome any feedback and ideas!
Let us know by submitting
\sphinxhref{https://github.com/BenjMy/agrogeophy-catalog/issues}{issues on Github}
or send us a message on our
\sphinxhref{https://agrogeophy.slack.com/}{Slack chatroom}.


\section{Citing}
\label{\detokenize{getting-started:citing}}
If you use agrogeophy\sphinxhyphen{}catalog for you work, please cite this paper as:

TODO

BibTex code:

\begin{sphinxVerbatim}[commandchars=\\\{\}]
\PYG{n}{TODO}
\end{sphinxVerbatim}


\chapter{CAGS \& FAIR principles}
\label{\detokenize{FAIR:cags-fair-principles}}\label{\detokenize{FAIR::doc}}
FAIR: Findability, Accessibility, Interoperability, and Reuse


\section{Findability}
\label{\detokenize{FAIR:findability}}
To enable an effective findability of users contributions, the website interface automatically extracts and inserts into the database all the metadata provided in the user submission form (for each type i.e. datasets, numerical studies and communications). The website interface allows then the retrieval and display of the contribution thanks to its search and filtering tool. A geospatial visualization and analysis component enables searching, visualizing, and analyzing users’ contributions.

\begin{figure}[htbp]
\centering
\capstart

\noindent\sphinxincludegraphics[scale=0.5]{{Map_contributions_26102020}.png}
\caption{Map of georeferenced studies (zoom on the Mediterranean basin). The map is based on the Leaflet framework and use map background from the OpenStreetMap project.}\label{\detokenize{FAIR:id1}}\label{\detokenize{FAIR:importing}}\end{figure}


\section{Accessibility}
\label{\detokenize{FAIR:accessibility}}
Once the user finds the required data, it can be access thanks to its DOI link. Metadata are accessible, even when the data are no longer available.


\section{Interoperability}
\label{\detokenize{FAIR:interoperability}}
The concept of standardized descriptive metadata provides a powerful mechanism to improve retrieval for specific applications and user communities and facilitates repository interoperability. The catalogue metadata architecture we implemented for CGAS has been inspired by the Archaeology Data Service (ADS, Richards, 2018) which acts as a metadata aggregator between archaeological and geophysical metadata.


\section{Reuse}
\label{\detokenize{FAIR:reuse}}
As a start, CAGS implement useful tools for the dataset:
\begin{quote}

To make this decision, the data publisher should provide not just metadata that allows discovery, but also metadata that richly describes the context under which the data was generated.
\end{quote}
\begin{itemize}
\item {} 
A data package linter , which could check for structure and files

\item {} 
At the raw data level, REDA may also serve as a translation service, converting old formats to new ones when necessary, especially for IP/SIP data, a fact that will highly contribute to the long\sphinxhyphen{}term data management and reuse.

\item {} 
At the (pre)\sphinxhyphen{}processed data level, the user may want to share the journal file produced using the REDA Python package.

\end{itemize}


\chapter{Glossary}
\label{\detokenize{glossary:glossary}}\label{\detokenize{glossary::doc}}

\section{Geoelectrical methods}
\label{\detokenize{glossary:geoelectrical-methods}}

\subsection{Graphical}
\label{\detokenize{glossary:graphical}}
\begin{figure}[htbp]
\centering
\capstart

\noindent\sphinxincludegraphics[scale=0.1]{{channels}.png}
\caption{Possible channels for soil/plant geoelectical measurements. Credit to L. Peruzzo \sphinxcite{glossary:cit2002}.}\label{\detokenize{glossary:id3}}\label{\detokenize{glossary:importing}}\end{figure}


\subsection{Acronyms}
\label{\detokenize{glossary:acronyms}}
\sphinxstylestrong{spectroscopic}
\begin{itemize}
\item {} 
EIS: Electrical Impedance Spectroscopy

\item {} 
IP: Induced Polarization

\item {} 
SIP: Spectral Induced Polarization

\item {} 
TDIP: Time\sphinxhyphen{}domain Induced Polarization

\end{itemize}

\sphinxstylestrong{imaging or tomography}
\begin{itemize}
\item {} 
ERT: Electrical Resistivity Tomography

\item {} 
ERI: Electrical Resistivity Imaging

\item {} 
ECT: Electrical Capacitance Tomography

\item {} 
EIT: Electrical Impedance Tomography

\item {} 
sEIT: spectral Electrical Impedance Tomography

\end{itemize}


\section{EM methods}
\label{\detokenize{glossary:em-methods}}

\subsection{Acronyms}
\label{\detokenize{glossary:id2}}\begin{itemize}
\item {} 
EMI: ElectroMagnetic Induction

\end{itemize}


\section{References}
\label{\detokenize{glossary:references}}

\chapter{The CAGS Metadata Schema}
\label{\detokenize{schema_documentation:the-cags-metadata-schema}}\label{\detokenize{schema_documentation::doc}}

\section{Project level metadata}
\label{\detokenize{schema_documentation:project-level-metadata}}
The \sphinxstylestrong{Project\sphinxhyphen{}level Metadata} level is related to the main CAGS interface website from which metadata are harvested through the online submission form. It collects metadata (derived and expended from Dublin Core metadata) which incorporates a number of descriptive and resource discovery focused elements describing either a scientific communication, a notebook and a dataset independently or jointly.

Tables below are extracted from the database and generated by MySQL Workbench Model Documentation v1.0.0 \sphinxhyphen{} Copyright (c)
2015 Hieu Le


\subsection{Table: \sphinxstyleliteralintitle{\sphinxupquote{abiotic}}}
\label{\detokenize{schema_documentation:table-abiotic}}

\subsubsection{Description:}
\label{\detokenize{schema_documentation:description}}

\subsubsection{Columns:}
\label{\detokenize{schema_documentation:columns}}

\begin{savenotes}\sphinxattablestart
\centering
\begin{tabulary}{\linewidth}[t]{|T|T|T|T|T|T|T|T|T|T|}
\hline

Column
&
Data type
&
Attributes
&
Default
&
Description
&
Multiplicity
&
Obligation
&
Metadata level
&
Dublin Core
compatibility
&
INSPIRE
compatibility
\\
\hline
id\_abiotic\_l2
&
SMALLINT
&
PRIMARY, Auto   increments, Not null
&&&&&
project
&&\\
\hline
soil\_type
&
VARCHAR(30)
&
Unique
&
NULL
&&
{[}1,n{]}
&
Optional
&
project
&
NONE
&
NONE
\\
\hline
water\_input
&
VARCHAR(30)
&
Unique
&
NULL
&
Triggered infiltration test:    Yes/No
&
{[}1,n{]}
&
Optional
&
project
&
NONE
&
NONE
\\
\hline
land\_use
&
VARCHAR(500)
&&
NULL
&&
{[}1,n{]}
&
Optional
&
project
&
NONE
&
NONE
\\
\hline
\end{tabulary}
\par
\sphinxattableend\end{savenotes}


\subsubsection{Indices:}
\label{\detokenize{schema_documentation:indices}}

\begin{savenotes}\sphinxattablestart
\centering
\begin{tabulary}{\linewidth}[t]{|T|T|T|T|}
\hline
\sphinxstyletheadfamily 
Name
&\sphinxstyletheadfamily 
Columns
&\sphinxstyletheadfamily 
Type
&\sphinxstyletheadfamily 
Description
\\
\hline
PRIMARY
&
\sphinxcode{\sphinxupquote{id\_abiotic\_l2}}
&
PRIMARY
&\\
\hline
un\_all\_abiotictable
&
\sphinxcode{\sphinxupquote{soil\_type}}, \sphinxcode{\sphinxupquote{water\_input}}
&
UNIQUE
&\\
\hline
\end{tabulary}
\par
\sphinxattableend\end{savenotes}


\subsection{Table: \sphinxstyleliteralintitle{\sphinxupquote{biotic}}}
\label{\detokenize{schema_documentation:table-biotic}}

\subsubsection{Description:}
\label{\detokenize{schema_documentation:id1}}

\subsubsection{Columns:}
\label{\detokenize{schema_documentation:id2}}

\begin{savenotes}\sphinxattablestart
\centering
\begin{tabulary}{\linewidth}[t]{|T|T|T|T|T|}
\hline
\sphinxstyletheadfamily 
Column
&\sphinxstyletheadfamily 
Data type
&\sphinxstyletheadfamily 
Attributes
&\sphinxstyletheadfamily 
Default
&\sphinxstyletheadfamily 
Description
\\
\hline
\sphinxcode{\sphinxupquote{id\_biotic\_l2}}
&
SMALLINT
&
PRIMARY, Auto increments, Not null
&&\\
\hline
\sphinxcode{\sphinxupquote{species}}
&
VARCHAR(30)
&
Unique
&
\sphinxcode{\sphinxupquote{NULL}}
&\\
\hline
\sphinxcode{\sphinxupquote{organ}}
&
VARCHAR(30)
&&
\sphinxcode{\sphinxupquote{NULL}}
&\\
\hline
\end{tabulary}
\par
\sphinxattableend\end{savenotes}


\subsubsection{Indices:}
\label{\detokenize{schema_documentation:id3}}

\begin{savenotes}\sphinxattablestart
\centering
\begin{tabulary}{\linewidth}[t]{|T|T|T|T|}
\hline
\sphinxstyletheadfamily 
Name
&\sphinxstyletheadfamily 
Columns
&\sphinxstyletheadfamily 
Type
&\sphinxstyletheadfamily 
Description
\\
\hline
PRIMARY
&
\sphinxcode{\sphinxupquote{id\_biotic\_l2}}
&
PRIMARY
&\\
\hline
un\_all\_biotictable
&
\sphinxcode{\sphinxupquote{species}}
&
UNIQUE
&\\
\hline
\end{tabulary}
\par
\sphinxattableend\end{savenotes}


\subsection{Table: \sphinxstyleliteralintitle{\sphinxupquote{contact}}}
\label{\detokenize{schema_documentation:table-contact}}

\subsubsection{Description:}
\label{\detokenize{schema_documentation:id4}}

\subsubsection{Columns:}
\label{\detokenize{schema_documentation:id5}}

\begin{savenotes}\sphinxattablestart
\centering
\begin{tabulary}{\linewidth}[t]{|T|T|T|T|T|}
\hline
\sphinxstyletheadfamily 
Column
&\sphinxstyletheadfamily 
Data type
&\sphinxstyletheadfamily 
Attributes
&\sphinxstyletheadfamily 
Default
&\sphinxstyletheadfamily 
Description
\\
\hline
\sphinxcode{\sphinxupquote{id\_contact\_l2}}
&
SMALLINT
&
PRIMARY, Auto increments, Not null
&&\\
\hline
\sphinxcode{\sphinxupquote{name}}
&
VARCHAR(30)
&
Unique
&
\sphinxcode{\sphinxupquote{NULL}}
&\\
\hline
\sphinxcode{\sphinxupquote{surname}}
&
VARCHAR(30)
&
Unique
&
\sphinxcode{\sphinxupquote{NULL}}
&\\
\hline
\sphinxcode{\sphinxupquote{organisation}}
&
VARCHAR(300)
&&
\sphinxcode{\sphinxupquote{NULL}}
&\\
\hline
\sphinxcode{\sphinxupquote{email}}
&
VARCHAR(60)
&
Unique
&
\sphinxcode{\sphinxupquote{NULL}}
&\\
\hline
\sphinxcode{\sphinxupquote{website\_perso}}
&
VARCHAR(300)
&&
\sphinxcode{\sphinxupquote{NULL}}
&\\
\hline
\end{tabulary}
\par
\sphinxattableend\end{savenotes}


\subsubsection{Indices:}
\label{\detokenize{schema_documentation:id6}}

\begin{savenotes}\sphinxattablestart
\centering
\begin{tabulary}{\linewidth}[t]{|T|T|T|T|}
\hline
\sphinxstyletheadfamily 
Name
&\sphinxstyletheadfamily 
Columns
&\sphinxstyletheadfamily 
Type
&\sphinxstyletheadfamily 
Description
\\
\hline
PRIMARY
&
\sphinxcode{\sphinxupquote{id\_contact\_l2}}
&
PRIMARY
&\\
\hline
un\_name\_surn\_email
&
\sphinxcode{\sphinxupquote{name}}, \sphinxcode{\sphinxupquote{surname}}, \sphinxcode{\sphinxupquote{email}}
&
UNIQUE
&\\
\hline
\end{tabulary}
\par
\sphinxattableend\end{savenotes}


\subsection{Table: \sphinxstyleliteralintitle{\sphinxupquote{main}}}
\label{\detokenize{schema_documentation:table-main}}

\subsubsection{Description:}
\label{\detokenize{schema_documentation:id7}}

\subsubsection{Columns:}
\label{\detokenize{schema_documentation:id8}}

\begin{savenotes}\sphinxattablestart
\centering
\begin{tabulary}{\linewidth}[t]{|T|T|T|T|T|}
\hline
\sphinxstyletheadfamily 
Column
&\sphinxstyletheadfamily 
Data type
&\sphinxstyletheadfamily 
Attributes
&\sphinxstyletheadfamily 
Default
&\sphinxstyletheadfamily 
Description
\\
\hline
\sphinxcode{\sphinxupquote{id}}
&
INT
&
PRIMARY, Auto increments, Not null
&&\\
\hline
\sphinxcode{\sphinxupquote{id\_FK\_contact}}
&
SMALLINT
&&
\sphinxcode{\sphinxupquote{NULL}}
&\\
\hline
\sphinxcode{\sphinxupquote{id\_FK\_prospection}}
&
SMALLINT
&&
\sphinxcode{\sphinxupquote{NULL}}
&\\
\hline
\sphinxcode{\sphinxupquote{id\_FK\_processing}}
&
SMALLINT
&&
\sphinxcode{\sphinxupquote{NULL}}
&\\
\hline
\sphinxcode{\sphinxupquote{id\_FK\_biotic}}
&
SMALLINT
&&
\sphinxcode{\sphinxupquote{NULL}}
&\\
\hline
\sphinxcode{\sphinxupquote{id\_FK\_abiotic}}
&
SMALLINT
&&
\sphinxcode{\sphinxupquote{NULL}}
&\\
\hline
\sphinxcode{\sphinxupquote{contrib\_type}}
&
VARCHAR(50)
&&
\sphinxcode{\sphinxupquote{NULL}}
&\\
\hline
\sphinxcode{\sphinxupquote{contrib\_title}}
&
VARCHAR(500)
&
Unique
&
\sphinxcode{\sphinxupquote{NULL}}
&\\
\hline
\sphinxcode{\sphinxupquote{contrib\_date}}
&
VARCHAR(50)
&&
\sphinxcode{\sphinxupquote{NULL}}
&\\
\hline
\sphinxcode{\sphinxupquote{contrib\_authors}}
&
VARCHAR(50)
&&
\sphinxcode{\sphinxupquote{NULL}}
&\\
\hline
\sphinxcode{\sphinxupquote{DOI}}
&
VARCHAR(50)
&
Unique
&
\sphinxcode{\sphinxupquote{NULL}}
&\\
\hline
\sphinxcode{\sphinxupquote{journal}}
&
VARCHAR(50)
&&
\sphinxcode{\sphinxupquote{NULL}}
&\\
\hline
\sphinxcode{\sphinxupquote{icon\_img}}
&
VARCHAR(500)
&&
\sphinxcode{\sphinxupquote{NULL}}
&\\
\hline
\sphinxcode{\sphinxupquote{keywords}}
&
VARCHAR(500)
&&
\sphinxcode{\sphinxupquote{NULL}}
&\\
\hline
\end{tabulary}
\par
\sphinxattableend\end{savenotes}


\subsubsection{Indices:}
\label{\detokenize{schema_documentation:id9}}

\begin{savenotes}\sphinxattablestart
\centering
\begin{tabulary}{\linewidth}[t]{|T|T|T|T|}
\hline
\sphinxstyletheadfamily 
Name
&\sphinxstyletheadfamily 
Columns
&\sphinxstyletheadfamily 
Type
&\sphinxstyletheadfamily 
Description
\\
\hline
PRIMARY
&
\sphinxcode{\sphinxupquote{id}}
&
PRIMARY
&\\
\hline
un\_FK\_main
&
\sphinxcode{\sphinxupquote{contrib\_title}}, \sphinxcode{\sphinxupquote{DOI}}
&
UNIQUE
&\\
\hline
id\_FK\_prospection
&
\sphinxcode{\sphinxupquote{id\_FK\_prospection}}
&
INDEX
&\\
\hline
id\_FK\_processing
&
\sphinxcode{\sphinxupquote{id\_FK\_processing}}
&
INDEX
&\\
\hline
id\_FK\_biotic
&
\sphinxcode{\sphinxupquote{id\_FK\_biotic}}
&
INDEX
&\\
\hline
id\_FK\_abiotic
&
\sphinxcode{\sphinxupquote{id\_FK\_abiotic}}
&
INDEX
&\\
\hline
id\_FK\_contact
&
\sphinxcode{\sphinxupquote{id\_FK\_contact}}
&
INDEX
&\\
\hline
\end{tabulary}
\par
\sphinxattableend\end{savenotes}


\subsection{Table: \sphinxstyleliteralintitle{\sphinxupquote{processing}}}
\label{\detokenize{schema_documentation:table-processing}}

\subsubsection{Description:}
\label{\detokenize{schema_documentation:id10}}

\subsubsection{Columns:}
\label{\detokenize{schema_documentation:id11}}

\begin{savenotes}\sphinxattablestart
\centering
\begin{tabulary}{\linewidth}[t]{|T|T|T|T|T|}
\hline
\sphinxstyletheadfamily 
Column
&\sphinxstyletheadfamily 
Data type
&\sphinxstyletheadfamily 
Attributes
&\sphinxstyletheadfamily 
Default
&\sphinxstyletheadfamily 
Description
\\
\hline
\sphinxcode{\sphinxupquote{id\_processing\_l2}}
&
SMALLINT
&
PRIMARY, Auto increments, Not null
&&\\
\hline
\sphinxcode{\sphinxupquote{software\_name}}
&
VARCHAR(30)
&
Unique
&
\sphinxcode{\sphinxupquote{NULL}}
&\\
\hline
\sphinxcode{\sphinxupquote{licence\_type}}
&
VARCHAR(30)
&
Unique
&
\sphinxcode{\sphinxupquote{NULL}}
&\\
\hline
\sphinxcode{\sphinxupquote{DOI\_software}}
&
VARCHAR(100)
&
Unique
&
\sphinxcode{\sphinxupquote{NULL}}
&\\
\hline
\sphinxcode{\sphinxupquote{notebook\_filename}}
&
VARCHAR(100)
&
Unique
&
\sphinxcode{\sphinxupquote{NULL}}
&\\
\hline
\sphinxcode{\sphinxupquote{notebook\_purpose}}
&
VARCHAR(100)
&
Unique
&
\sphinxcode{\sphinxupquote{NULL}}
&\\
\hline
\sphinxcode{\sphinxupquote{data\_repo\_url}}
&
VARCHAR(100)
&
Unique
&
\sphinxcode{\sphinxupquote{NULL}}
&\\
\hline
\sphinxcode{\sphinxupquote{data\_licence}}
&
VARCHAR(100)
&
Unique
&
\sphinxcode{\sphinxupquote{NULL}}
&\\
\hline
\end{tabulary}
\par
\sphinxattableend\end{savenotes}


\subsubsection{Indices:}
\label{\detokenize{schema_documentation:id12}}

\begin{savenotes}\sphinxattablestart
\centering
\begin{tabulary}{\linewidth}[t]{|T|T|T|T|}
\hline
\sphinxstyletheadfamily 
Name
&\sphinxstyletheadfamily 
Columns
&\sphinxstyletheadfamily 
Type
&\sphinxstyletheadfamily 
Description
\\
\hline
PRIMARY
&
\sphinxcode{\sphinxupquote{id\_processing\_l2}}
&
PRIMARY
&\\
\hline
un\_all\_proctable
&
\sphinxcode{\sphinxupquote{software\_name}}, \sphinxcode{\sphinxupquote{licence\_type}}, \sphinxcode{\sphinxupquote{DOI\_software}}, \sphinxcode{\sphinxupquote{notebook\_filename}}, \sphinxcode{\sphinxupquote{notebook\_purpose}}, \sphinxcode{\sphinxupquote{data\_repo\_url}}, \sphinxcode{\sphinxupquote{data\_licence}}
&
UNIQUE
&\\
\hline
\end{tabulary}
\par
\sphinxattableend\end{savenotes}


\subsection{Table: \sphinxstyleliteralintitle{\sphinxupquote{prospection}}}
\label{\detokenize{schema_documentation:table-prospection}}

\subsubsection{Description:}
\label{\detokenize{schema_documentation:id13}}

\subsubsection{Columns:}
\label{\detokenize{schema_documentation:id14}}

\begin{savenotes}\sphinxattablestart
\centering
\begin{tabulary}{\linewidth}[t]{|T|T|T|T|T|}
\hline
\sphinxstyletheadfamily 
Column
&\sphinxstyletheadfamily 
Data type
&\sphinxstyletheadfamily 
Attributes
&\sphinxstyletheadfamily 
Default
&\sphinxstyletheadfamily 
Description
\\
\hline
\sphinxcode{\sphinxupquote{id\_prospection\_l2}}
&
SMALLINT
&
PRIMARY, Auto increments, Not null
&&\\
\hline
\sphinxcode{\sphinxupquote{datep}}
&
DATE
&
Unique
&
\sphinxcode{\sphinxupquote{NULL}}
&\\
\hline
\sphinxcode{\sphinxupquote{lat}}
&
DOUBLE
&
Unique
&
\sphinxcode{\sphinxupquote{NULL}}
&\\
\hline
\sphinxcode{\sphinxupquote{longitude}}
&
DOUBLE
&
Unique
&
\sphinxcode{\sphinxupquote{NULL}}
&\\
\hline
\sphinxcode{\sphinxupquote{method}}
&
VARCHAR(30)
&
Unique
&
\sphinxcode{\sphinxupquote{NULL}}
&\\
\hline
\sphinxcode{\sphinxupquote{spatial\_scale}}
&
VARCHAR(30)
&&
\sphinxcode{\sphinxupquote{NULL}}
&\\
\hline
\sphinxcode{\sphinxupquote{bound\_cond}}
&
VARCHAR(30)
&&
\sphinxcode{\sphinxupquote{NULL}}
&\\
\hline
\sphinxcode{\sphinxupquote{temperature}}
&
VARCHAR(30)
&&
\sphinxcode{\sphinxupquote{NULL}}
&\\
\hline
\sphinxcode{\sphinxupquote{temporal\_scale}}
&
VARCHAR(30)
&&
\sphinxcode{\sphinxupquote{NULL}}
&\\
\hline
\sphinxcode{\sphinxupquote{instrument}}
&
VARCHAR(30)
&
Unique
&
\sphinxcode{\sphinxupquote{NULL}}
&\\
\hline
\sphinxcode{\sphinxupquote{dimension}}
&
VARCHAR(30)
&&
\sphinxcode{\sphinxupquote{NULL}}
&\\
\hline
\sphinxcode{\sphinxupquote{permanent\_setup}}
&
VARCHAR(30)
&&
\sphinxcode{\sphinxupquote{NULL}}
&\\
\hline
\sphinxcode{\sphinxupquote{zhao\_description}}
&
VARCHAR(300)
&&
\sphinxcode{\sphinxupquote{NULL}}
&\\
\hline
\end{tabulary}
\par
\sphinxattableend\end{savenotes}


\subsubsection{Indices:}
\label{\detokenize{schema_documentation:id15}}

\begin{savenotes}\sphinxattablestart
\centering
\begin{tabulary}{\linewidth}[t]{|T|T|T|T|}
\hline
\sphinxstyletheadfamily 
Name
&\sphinxstyletheadfamily 
Columns
&\sphinxstyletheadfamily 
Type
&\sphinxstyletheadfamily 
Description
\\
\hline
PRIMARY
&
\sphinxcode{\sphinxupquote{id\_prospection\_l2}}
&
PRIMARY
&\\
\hline
un\_all\_prosptable
&
\sphinxcode{\sphinxupquote{datep}}, \sphinxcode{\sphinxupquote{lat}}, \sphinxcode{\sphinxupquote{longitude}}, \sphinxcode{\sphinxupquote{method}}, \sphinxcode{\sphinxupquote{instrument}}
&
UNIQUE
&\\
\hline
\end{tabulary}
\par
\sphinxattableend\end{savenotes}


\section{File\sphinxhyphen{}level geophysical metadata}
\label{\detokenize{schema_documentation:file-level-geophysical-metadata}}
The \sphinxstylestrong{File\sphinxhyphen{}level Metadata} level collector is a gui designed to help with the initial preparation of one geophysical dataset. Starting from one or multiple input directories, a cleanly structured output directory is generated (without deleting any input files).
Generate suitable metadata from user input and write this metadata into the directory structure, making it ready for further distribution.

\sphinxhref{https://github.com/m-weigand/geometadp.git}{geophysical Metadata Management using a Juypter Notebook}

Tables below are extracted from the database and generated by MySQL Workbench Model Documentation v1.0.0 \sphinxhyphen{} Copyright (c)
2015 Hieu Le


\subsection{Table: \sphinxstyleliteralintitle{\sphinxupquote{report}}}
\label{\detokenize{schema_documentation:table-report}}

\subsubsection{Description:}
\label{\detokenize{schema_documentation:id16}}
Metadata describing general information about the contribution.

This table has overlapping field with project level metadata (see column
CAGS Metadata level)


\subsubsection{Columns:}
\label{\detokenize{schema_documentation:id17}}

\begin{savenotes}\sphinxattablestart
\centering
\begin{tabulary}{\linewidth}[t]{|T|T|T|T|T|T|T|T|T|T|}
\hline
\sphinxstyletheadfamily 
Column
&\sphinxstyletheadfamily 
Description
&\sphinxstyletheadfamily 
Multiplicity
&\sphinxstyletheadfamily 
Obligation
&\sphinxstyletheadfamily 
CAGS Metadata level
&\sphinxstyletheadfamily 
Dublin Core (ArchSearch)
&\sphinxstyletheadfamily 
INSPIRE Directive
&\sphinxstyletheadfamily 
Data type
&\sphinxstyletheadfamily 
Attributes
&\sphinxstyletheadfamily 
Default
\\
\hline
Report\_title
&
Sort title description of the dataset
&
{[}1{]}
&
Mandatory
&
File level
&&&&
Unique
&\\
\hline
Report\_author
&
Reporting authors names
&
{[}1,n{]}
&
Mandatory
&
File \& project level
&&&&
Unique
&\\
\hline
\end{tabulary}
\par
\sphinxattableend\end{savenotes}


\subsection{Table: \sphinxstyleliteralintitle{\sphinxupquote{survey}}}
\label{\detokenize{schema_documentation:table-survey}}

\subsubsection{Description:}
\label{\detokenize{schema_documentation:description-1}}\label{\detokenize{schema_documentation:id18}}
Metadata describing one to multiple survey(s).

The survey table is inspired from \textasciigrave{}Archaeology Data Service / Digital
Antiquity

Guides to Good Practice
\textless{}\sphinxurl{https://guides.archaeologydataservice.ac.uk/g2gp/Geophysics\_6}\textgreater{}\textasciigrave{}\_

For multiple acquisitions the number n must be unchanged between the
different fields. For example, if date of time of measurement contains 2
values, the electrode configuration must contain n columns describing
the configuration used.


\subsubsection{Columns:}
\label{\detokenize{schema_documentation:columns-1}}\label{\detokenize{schema_documentation:id19}}

\begin{savenotes}\sphinxattablestart
\centering
\begin{tabulary}{\linewidth}[t]{|T|T|T|T|T|T|T|T|T|T|}
\hline
\sphinxstyletheadfamily 
Column
&\sphinxstyletheadfamily 
Description
&\sphinxstyletheadfamily 
Multiplicity
&\sphinxstyletheadfamily 
Obligation
&\sphinxstyletheadfamily 
CAGS

Metadata   level
&\sphinxstyletheadfamily 
Dublin   Core (ArchSearch)
&\sphinxstyletheadfamily 
INSPIRE   Directive
&\sphinxstyletheadfamily 
Data   type
&\sphinxstyletheadfamily 
Attributes
&\sphinxstyletheadfamily 
Default
\\
\hline
Survey\_type
&
Choose acronyms describing the survey   type (refer to CAGS glossary)
&
{[}1,n{]}
&
Mandatory
&
File level
&
Resource Type
&&&
Unique
&\\
\hline
Instruments
&
Name of the instrument(s)
&
{[}1,n{]}
&
Mandatory
&
File level
&&&&&
NULL
\\
\hline
Choice\_survey
&
Text explaining shortly the motivation of   using the method(s)
&
{[}1,n{]}
&
Optional
&
File level
&&&&&
NULL
\\
\hline
Area
&
Total surface investigated (m2)
&
{[}1,n{]}
&
Optional
&
File level
&&&&&
NULL
\\
\hline
Add\_remarks
&
Free text for additional remarks
&
{[}1{]}
&
Optional
&
File level
&&&&&
NULL
\\
\hline
\end{tabulary}
\par
\sphinxattableend\end{savenotes}


\subsection{Table: \sphinxstyleliteralintitle{\sphinxupquote{ERT metadata}}}
\label{\detokenize{schema_documentation:table-ert-metadata}}

\subsubsection{Description:}
\label{\detokenize{schema_documentation:description-2}}\label{\detokenize{schema_documentation:id20}}
Metadata describing one to multiple (n) ERT surveys.

For multiple acquisitions the number n must be unchanged between the
different fields. For example, if date of time of measurement contains 2
values, the electrode configuration must contain n columns describing
the configuration used.


\subsubsection{Columns:}
\label{\detokenize{schema_documentation:columns-3}}\label{\detokenize{schema_documentation:id21}}

\begin{savenotes}\sphinxattablestart
\centering
\begin{tabulary}{\linewidth}[t]{|T|T|T|T|T|T|T|T|T|T|}
\hline
\sphinxstyletheadfamily 
Column
&\sphinxstyletheadfamily 
Description
&\sphinxstyletheadfamily 
Multiplicity
&\sphinxstyletheadfamily 
Obligation
&\sphinxstyletheadfamily 
CAGS

Metadata level
&\sphinxstyletheadfamily 
Dublin Core (ArchSearch)
&\sphinxstyletheadfamily 
INSPIRE Directive
&\sphinxstyletheadfamily 
Data type
&\sphinxstyletheadfamily 
Attributes
&\sphinxstyletheadfamily 
Default
\\
\hline
Date\_measure
&
Date(s) of the measurement (dd/mm/aaaa)
&
{[}n{]}
&
Mandatory
&
File level
&&&&
Unique
&\\
\hline
Time\_measure
&
Time(s) of the measurement (hh:mm)
&
{[}m,n{]}
&
Mandatory
&
File level
&&&&&
NULL
\\
\hline
Elec\_conf
&
Electrode   configuration
&
{[}m,n{]}
&
Optional
&
File   level
&&&&&
NULL
\\
\hline
Elec\_spacing
&
Electrode spacing
&
{[}m,n{]}
&
Optional
&
File level
&&&&&
NULL
\\
\hline
\end{tabulary}
\par
\sphinxattableend\end{savenotes}


\subsection{Table: \sphinxstyleliteralintitle{\sphinxupquote{EM metadata}}}
\label{\detokenize{schema_documentation:table-em-metadata}}

\subsubsection{Description:}
\label{\detokenize{schema_documentation:description-3}}\label{\detokenize{schema_documentation:id22}}
Metadata describing one to multiple (n) EM surveys.

For multiple acquisitions the number n must be unchanged between the
different fields. For example, if date of measurements contains 2
values, the coil configuration must contain n columns and m lines
describing the coil configuration used.


\subsubsection{Columns:}
\label{\detokenize{schema_documentation:columns-4}}\label{\detokenize{schema_documentation:id23}}

\begin{savenotes}\sphinxattablestart
\centering
\begin{tabulary}{\linewidth}[t]{|T|T|T|T|T|T|T|T|T|T|}
\hline
\sphinxstyletheadfamily 
Column
&\sphinxstyletheadfamily 
Description
&\sphinxstyletheadfamily 
Multiplicity
&\sphinxstyletheadfamily 
Obligation
&\sphinxstyletheadfamily 
CAGS

Metadata level
&\sphinxstyletheadfamily 
Dublin Core (ArchSearch)
&\sphinxstyletheadfamily 
INSPIRE Directive
&\sphinxstyletheadfamily 
Data type
&\sphinxstyletheadfamily 
Attributes
&\sphinxstyletheadfamily 
Default
\\
\hline
Date\_measure
&
Date(s) of the measurement (dd/mm/aaaa)
&
{[}1,n{]}
&
Mandatory
&
File level
&&&&
Unique
&\\
\hline
Coil\_conf
&
Coil configuation
&
{[}m,n{]}
&
Optional
&
File level
&&&&&
NULL
\\
\hline
Read\_interval
&
If automatic time sampling, time steps   between different reading
&
{[}m,n{]}
&
Optional
&
File level
&&&&&
NULL
\\
\hline
\end{tabulary}
\par
\sphinxattableend\end{savenotes}


\subsection{Table: \sphinxstyleliteralintitle{\sphinxupquote{data quality assessment metadata}}}
\label{\detokenize{schema_documentation:table-data-quality-assessment-metadata}}

\subsubsection{Description:}
\label{\detokenize{schema_documentation:description-4}}\label{\detokenize{schema_documentation:id24}}

\subsubsection{Columns:}
\label{\detokenize{schema_documentation:columns-5}}\label{\detokenize{schema_documentation:id25}}

\begin{savenotes}\sphinxattablestart
\centering
\begin{tabulary}{\linewidth}[t]{|T|T|T|T|T|T|T|T|T|T|}
\hline
\sphinxstyletheadfamily 
Column
&\sphinxstyletheadfamily 
Description
&\sphinxstyletheadfamily 
Multiplicity
&\sphinxstyletheadfamily 
Obligation
&\sphinxstyletheadfamily 
CAGS

Metadata level
&\sphinxstyletheadfamily 
Dublin Core (ArchSearch)
&\sphinxstyletheadfamily 
INSPIRE Directive
&\sphinxstyletheadfamily 
Data type
&\sphinxstyletheadfamily 
Attributes
&\sphinxstyletheadfamily 
Default
\\
\hline
Peer\_reviewed
&
True if the dataset has been   peer\sphinxhyphen{}reviewed
&
{[}1{]}
&
Mandatory
&
File level
&&&&
Unique
&\\
\hline
Peer\_reviewer\_contact
&
Contact of reviewer
&
{[}1,n{]}
&
Mandatory only if peer\_reviewed is True
&&&&&
Unique
&
NULL
\\
\hline
Replicate\_datasets
&
Number of replicates datasets
&
{[}1,n{]}
&
Optional
&&&&&&
NULL
\\
\hline
Comparison\_ref\_data
&
The dataset has been compared with   reference datasets
&
{[}1,n{]}
&
Optional
&&&&&&
NULL
\\
\hline
Ref\_data
&
DOI of reference dataset
&
{[}1,n{]}
&
Optional
&&&&&&
NULL
\\
\hline
\end{tabulary}
\par
\sphinxattableend\end{savenotes}


\subsection{Table: \sphinxstyleliteralintitle{\sphinxupquote{sampling}}}
\label{\detokenize{schema_documentation:table-sampling}}\label{\detokenize{schema_documentation:section-1}}

\subsubsection{Description:}
\label{\detokenize{schema_documentation:description-5}}\label{\detokenize{schema_documentation:id26}}

\subsubsection{Columns:}
\label{\detokenize{schema_documentation:columns-6}}\label{\detokenize{schema_documentation:id27}}\phantomsection\label{\detokenize{schema_documentation:section-2}}

\chapter{Indices and tables}
\label{\detokenize{index:indices-and-tables}}\begin{itemize}
\item {} 
\DUrole{xref,std,std-ref}{genindex}

\item {} 
\DUrole{xref,std,std-ref}{modindex}

\item {} 
\DUrole{xref,std,std-ref}{search}

\end{itemize}

\begin{sphinxthebibliography}{CIT2002}
\bibitem[CIT2002]{glossary:cit2002}
Peruzzo, L., Chou, C., Wu, Y. et al. Imaging of plant current pathways for non\sphinxhyphen{}invasive root Phenotyping using a newly developed electrical current source density approach. Plant Soil 450, 567\textendash{}584 (2020). \sphinxurl{https://doi.org/10.1007/s11104-020-04529-w}
\end{sphinxthebibliography}



\renewcommand{\indexname}{Index}
\printindex
\end{document}